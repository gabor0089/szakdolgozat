\documentclass[
% opciók nélkül: egyoldalas nyomtatás, elektronikus verzió (fekete linkek)
% twoside,     % kétoldalas nyomtatás
% colorlinks,  % elektronikus verzió (színes linkek)
% tocnopagenum,% oldalszámozás a tartalomjegyzék után kezdődik
]{thesis-ekf}
\usepackage[T1]{fontenc}
\PassOptionsToPackage{defaults=hu-min}{magyar.ldf}
\usepackage[magyar]{babel}
\footnotestyle{rule=fourth}
\usepackage{natbib}
\setcitestyle{aysep={}}

\begin{document}
\institute{Matematikai és Informatikai Intézet}
\title{Középiskolai e-napló fejlesztése MVC keretrendszer használatával}
\author{Renyhárt Gábor\\programtervező informatikus BSc}
\supervisor{Balla Tamás\\tanársegéd}
\city{Eger}
\date{2020}
\maketitle

\tableofcontents

\chapter*{Bevezetés}
Mindannyian voltunk diákok, jártunk iskolába, és mindenki kapott jegyet. Az én időmben még papír alapon. A Naplóba. Már általános iskola végén sem értettem, hogy ha ennyi számítógép van már a világban, akkor miért nem használják egy ilyen területen, ahol sok az adminisztráció, és sok a hibázási lehetőség. Az évek során a tanulmányaim folytán jutottam el arra a szintre, hogy képes vagyok egy ilyen alkalmazás megírására.
Bízom benne, hogy hamarosan már mindenhol csak elektronikus naplót fognak használni, és miért is ne, esetleg pont ez a szakdolgozat lesz a jövő középiskolai elektronikus rendszerének az alapja.


\chapter{Probléma kifejtése}
A jelenleg használatban lévő e-naplók a vélemények szerint bonyolultak, sok időt vesznek el az oktatóktól, az adminisztrátoroktól. Viszont nincsenek olyannyira kihasználva, amennyire a XXI. századi diákoknak szüksége lenne rá. A célom, hogy egy olyan alkalmazást készítsek el, ami kellően egyszerűen, gyorsan használható és szinte tökéletes.
\chapter{Implementálás}
A forráskódot a Sublime Text 3 programmal írtam. Az első tesztekhez a saját számítógépemen WampServer-t használok, mely a Windows-Apache-Mysql-PHP szavakból tevődik össze. Az adatbázisban az adattáblák harmadik normálformában vannak, 19 tábla összekapcsolásából áll össze az alkalmazás.
\chapter{Problémák, és megoldásuk a fejlesztés során}
Ebben a fejezetben kifejtem, hogy az alkalmazás mely részénél futottam bele olyan problémákba, amit nem tudtam egyszerűen, könnyen átlépni. Először a probléma felmerülésének helyét, leírását adom meg, majd több megoldást kipróbálok, végül leírom miért pont az adott megoldás lett a tökéletes.
\chapter{Felhasználói dokumentáció}
Ebben a fejezetben a felhasználók számára, világosan, érthetően, képernyőképekkel illusztrálva írom le a program működésének minden olyan funkcióját, amire szükségük lehet az alkalmazás használata során.
\chapter{Továbbfejlesztési lehetőségek}
Szeretnék egy olyan rendszert megalkotni, ami már-már továbbfejleszthetetlen. De természetesen ilyen nem létezik, soha nem is fog. Minden programban vannak továbbfejlesztési lehetőségek.
\section{Órarend}
Egy elektronikus naplóban kézenfekvő megoldás, ha az órarendet is már az alkalmazás készíti el. Ahhoz, hogy ezt el lehessen készíteni, szükség van a tanárok, az osztályok, és a tantermek rögzítésére. Természetesen nem megfeledkezve a tantárgyakról, ami minden osztály esetében más és más. Ezek az adatok az általam készített programokban természetesen megvannak, így már csak maga az órarend generálásra lenne hátra. Ennek a megalkotása viszont felérne egy újabb fél éves fejlesztési folyamattal.
\section{Házi feladatok}
Ugyanúgy, ahogyan a jelenlegi programban a dolgozatok meg vannak valósítva, meg lehet ugyanezt a házi feladatra is valósítani. A  tanár kiadja a feladatot online, és a diákok otthon megcsinálva, feltölthetik a naplóba, ahonnan a tanár letöltés, nyomtatás, és piros toll használata nélkül egyszerűen tudná ellenőrizni. 
\chapter{Összefoglalás}
Ebbe a fejezetbe kerül a probléma ismételt leírása, valamint a megoldásával együtt egy következtetés levonás.
\begin{thebibliography}{}
	https://thispersondoesnotexist.com/ \\
	https://codingislove.com/realtime-search-javascript/
\end{thebibliography}
\end{document}